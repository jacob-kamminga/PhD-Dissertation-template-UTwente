\section{Introduction}
\label{sec:intro_introduction}

The ability to automatically recognize animal activity from sensor data may be essential for their survival.
In 2018, Bar-On et al.~\cite{BarOn2018} estimated that out of all living mammals on Earth \SI{60}{\%} are livestock, \SI{36}{\%} humans, and only  \SI{4}{\%} are wild animals.
In a relatively short period, human development of civilization caused a loss of \SI{83}{\%} of all wildlife and \SI{50}{\%} of all plants.
Moreover, the current rate of the global decline in wildlife populations is unprecedented in human history -- and the rate of species extinctions is accelerating~\cite{Diaz2019}. 
Throughout 2016, a~rhino was killed every 8 hours for its horn in South Africa alone~\cite{PoachingStat}.
Moreover, an~elephant is killed every 20 minutes~\cite{Wittemyer2014}.
Simultaneously, the human population keeps growing, and as living standards in developing countries improve, the demand for meat and dairy products increases~\cite{Gandhi2010}.
A better insight into the well-being of livestock is not only an opportunity to improve their lives~\cite{Mench1998} but also an opportunity to help farmers produce more efficiently~\cite{Lokhorst2018}.
Ultimately, understanding the ecology of wild animals provides critical insight that allows world politics to make better choices to guarantee their survival.
It is vital to study the ecology of wildlife for their conservation and the well-being of livestock to improve their lives.




\subsection{Chapter Organization}

The rest of the chapter is organized as follows.
In the following section we discuss various application domains that are supported by \gls{aar}.
In \sectionname~\ref{sec:intro_challenges}, we discuss the challenges and requirements of \gls{aar}.
Subsequently, we discuss our research objective, questions, and hypotheses in \sectionname~\ref{sec:intro_research_objective}.
We present our contributions that address the research questions in \sectionname~\ref{sec:intro_contributions}.
Finally, we discuss the organization of the remainder of this thesis in \sectionname~\ref{sec:intro_thesis_organization}.