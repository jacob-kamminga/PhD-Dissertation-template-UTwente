\begin{table}[htbp]
	\centering
	\caption{{Overview of poaching detection technologies.}}
	\label{tab:approach_comparison}%
	%Please define any abbreviations used in the Table directly below the Table if appropriate
	% Table generated by Excel2LaTeX from sheet 'approach_comparison'
		\resizebox{\linewidth}{!}{%
\begin{tabular}{p{18.335em}p{19.28em}p{22.28em}p{10.055em}}
\toprule
\multicolumn{1}{c}{\textbf{Technique}} & \multicolumn{1}{c}{\textbf{Advantages}} & \multicolumn{1}{c}{\textbf{Disadvantages}} & \multicolumn{1}{c}{\textbf{Ref.}} \\
\midrule
\textbf{Perimeter Based Technologies} & Fences are often already in place (sometimes~electrified) and can be fortified with the surveyed approaches; some of the surveyed approaches are commercially~available & Detect intrusion only along the perimeter of an~area, not inside the area itself (linear detection zone). Poachers can enter through the main gate, e.g.,~disguised as tourist operators. & \\
       Lasers combined with movement detection PIR sensors & Lasers can cover larger distances & No classification; triggered by plants and animals; large \gls{far} & \citep{Cambron2015} \\
       Sensor nodes with accelerometers attached to a~fence & Classification of intrusion event, thus lower~\gls{far} & Many sensors needed; low~stealth & \citep{Wittenburg2007,Yousefi2008} \\
       Microphonic cables attached to fence & Classification and localization of intrusion & 200 m segments; a~lot of infrastructure needed; low~stealth & \citep{Stellar,Maki2007,Maki1998,Microwave,Backx2002} \\
       Optical fiber attached to fence & Classification and localization of intrusion; segments up to 1000 m; no power needed along segments; insensitive to electromagnetic inference; very sensitive; reliable & Expensive; low~stealth. & \citep{Maki2007} \\
       Buried optical fiber to detect footsteps & High stealth; harder to destroy; segment ranges up to 10 km & Difficult to bury cables in wildlife areas; soil types vary; expensive & \citep{Mishra2013,Snider} \\
       Networked sensors of various types (infrared, magnetic, camera) on and around a~fence & Higher resilience; some works include distributed algorithms that aim to decrease~\gls{far} & Many sensors needed; large overhead & \citep{Dziengel2015,Kim2008,Kim2005,Rothenpieler2009,Aseeri2014,He2004,He2006,Sun2011} \\
\midrule
{\textbf{Ground Based Technologies}} & Can detect intruders on larger area; not~limited to linear zone. & Any infrastructure placed inside a~wildlife area is prone to be damaged by wildlife. & \\
       Buried coaxial cable & High stealth. Field is wider than optical fiber approach. Commercially available. & Difficult to bury cables in wildlife areas; soil types vary; expensive; volumetric range is not very high. & \citep{harman1976guidar,Poirier1982,Harman2005, Inomata2014,Inomata2006} \\
       Fixed sensor node placement with various sensors (RADAR, microphone, light~intensity, magnetometers) & Improved animal tracking & Many sensors needed; deployment difficulties such as power usage and destruction of nodes & \citep{Souza2015,Mishra2010,Arora2004,He2014} \\
       Recording animal sounds & Some animals can be heard 4 km away; thus~larger range & More challenging approach because: necessary~to understand animal sounds; different acoustic characteristics are found in different environments; difference in the vocal repertoire between~different~species & \citep{Zeppelzauer2015} \\
       Gunshot detection & Gunshots can be heard from far away; thus~larger range & High chance of animal being killed before poacher~detection & \citep{Jumnani2014} \\
       Ultra Wide Band (UWB) WSN & Classification of intrusion event; thus lower \gls{far}; higher stealth. Improved detection in forested areas. & Limited range; many sensors needed; deployment difficulties such as power usage and destruction of~nodes & \citep{Zhang2014a,Jiang2014} \\
\midrule
{\textbf{Aerial Based Technologies}} & Very agile and can cover large areas & Aerial based techniques are obtrusive to habitants and tourists; crashing drones can be a~hazard to people and wildlife; can be vulnerable to shooting & \\
       Drones with heat sensing and camera equipment & Works up to 180 m height, thus large range & Unable to detect people under foliage; high~running~costs & \citep{Mulero-Pazmany2014,Zhang2014,wcuavc_web} \\
       Using predictive analytics for automated air surveillance & Improved surveillance accuracy; less sensors needed & Unable to detect people under foliage; high~running~costs & \citep{NoseongPark2015, Park2016} \\
\midrule
{\textbf{Animal Tagging Technologies}} & Can potentially cover very large areas with high sensitivity & Many sensors needed; deployment difficulties such as power usage and collaring &  \\
       Attach various sensors (cameras, motion, GPS) to animals and classify anomalous behavior & Timely notification of anomalies & Difficult to classify anomalies & \citep{Sahin2007,FirmatBanzi2014,Recio2011} \\
       Monitor physiological status of rhino and implement camera + GPS in rhino horn & Timely notification of animal distress or death; possibility to identify poachers through photos taken from the horn & Still high chance of animal being killed; location data of rhinos is very valuable and can motivate corruption & \citep{o2016real,ProjectRAPID} \\
       Detect horn separation from body through~RFID & Helps to notify rangers as soon as animal is poached and increases possibility of the poacher's capture & Rhino will be killed; the RFID chips will grow out of the horn & \citep{intel_rhino_chip} \\ 
\bottomrule
\end{tabular} 
}
%@reply: added legend to table

\medskip
	
\footnotesize PIR, passive infrared; FAR, false alarm rate; RADAR, radio detection and ranging; UWB, ultra wide band; GPS, global positioning system; RFID, radio-frequency identification.

	
\end{table}%