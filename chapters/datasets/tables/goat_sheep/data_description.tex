% Table generated by Excel2LaTeX from sheet 'Sheet1'
\begin{table}[htb]
  \centering
  \caption{Column description of Datasets 1 and 2}
% Table generated by Excel2LaTeX from sheet 'Sheet1'
% Table generated by Excel2LaTeX from sheet 'Sheet1'
\begin{tabular}{lp{35em}}
\toprule
\multicolumn{1}{p{6.645em}}{\textbf{Column name}} & \textbf{Description} \\
\midrule
ax    & Raw data from accelerometer x-axis \\
ay    & Raw data from accelerometer y-axis \\
az    & Raw data from accelerometer z-axis \\
axhg  & Raw data from high G accelerometer x-axis. \\
ayhg  & Raw data from high G accelerometer y-axis. \\
azhg  & Raw data from high G accelerometer z-axis. \\
gx    & Raw data from gyroscope x-axis \\
gy    & Raw data from gyroscope y-axis \\
gz    & Raw data from gyroscope z-axis \\
cx    & Raw data from compass (magnetometer) x-axis \\
cy    & Raw data from compass (magnetometer) y-axis \\
cz    & Raw data from compass (magnetometer) z-axis \\
pressure & Raw data from barometer \\
temp  & Raw data from temperature meter \\
label & Label that belongs to each row's data \\
animal\_ID & Subject identifier. 'S1' = Sheep 1 and 'G1' = Goat 1 etc. \\
segment\_ID & Each activity has been sorted per segment. Data within one segment is continuous. Segments have been numbered incrementally for each animal and are not consecutive \\
timestamp\_ms & The data for the accelerometers and gyroscopes have been recorded at 200 Hz. Therefore the timestamp column increments per 5ms. Please note that although the timestamp column seems consecutive over\newline{}all segments, the segments are not consecutive. Sensors that had a lower sampling rate have missing values with 'NaN' value \\
\bottomrule
\end{tabular}%


  \label{tab:goat_sheep_data_description}%
\end{table}%





