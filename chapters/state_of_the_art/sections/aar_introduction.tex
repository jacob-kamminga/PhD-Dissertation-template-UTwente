\section{Introduction}


The behavior of animals is a rich source of information that not only provides insights into their life and well-being but also their environment~\cite{Sahin2007,Petersen2002,FirmatBanzi2014,Nathan2008}.
Biologists have attempted to study the behavior of animals using technology for decades.
In the 1960s, biologists attached balloons and kitchen timers to marine mammals in order to study their movements and diving physiology~\cite{Wilmers2015}. 
In recent years, there has been a considerable rise in interest in activity monitoring of animals using sensors and embedded devices. 
The advent of small, lightweight, and low-power electronics has made it possible to monitor and detect animal activities -- in real-time -- on the animal itself and has propelled research in~\gls{aar}.

\subsection{Chapter Organization}

The rest of the chapter is organized as follows.
In the following section we first discuss the aforementioned steps in the \gls{aar} pipeline in more detail and explain how these have been approached in the state of the art.
Subsequently, we survey and compare various \gls{aar} studies in \sectionname~\ref{sec:stateoftheart_survey}. 

